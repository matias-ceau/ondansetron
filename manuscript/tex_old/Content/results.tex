
%\input{tables/table}
%%%%%%%%%%%%%%%%%%%%%%%%%%%%%%%%%%%%%%%%%%%%%%%%

\paragraph{Study selection}
A total of 90 results were found in the MEDLINE database, 68 on PsycInfo, 35 on Cochrane Library and 8 additional studies were found on Wiley Online Library.  

A total of 134 articles were identified through the search of the databases. After review of titles, 40 articles were selected for further examination.

After reading the full text, 21 met inclusion criteria for this review. This process is described in the PRISMA flowchart (Figure~\ref{fig:flowchart}). 

Among the 21 included studies, 18 were randomized controlled trials (RCT) or analysis of previous RCT, and 3 were prospective open-label studies (\cite{kranzler_effects_2003}, \cite{dawes_prospective_2005}, \cite{dawes_reductions_2005}). The study duration ranged from 2 to 12 weeks. The selected articles were published between 1994 and 2015. 

\paragraph{Quality assessment and risk of bias}
Randomized controlled trials where analyzed using the Cochrane Rob2 tool (see Figure~\ref{fig:trafficlight} for the traffic-light plot). Two studies had a low concern of bias (\cite{sellers_clinical_1994}, \cite{myrick_effect_2008}), most of the studies showed some concern of bias and one had a high concern (\cite{correa_filho_pilot_2013}).
The high dropout rate of the studies was the most concerning factor and affected the Domain 3 of the Risk of Bias tool which represent bias due to missing outcome result.
For the 3 prospective study, the Newcastle-Ottawa scale was used (see Figure~\ref{tab:nos}). The risk of bias was evaluated as acceptable as one study scored 7 out of 9 possible points and the two others scored 6 out of 7 (adapted score).

\subsubsection*{Study results}
Results are presented according to their primary outcomes. Among the 21 included studies, 17 directly assessed ondansetron efficacy through alcohol use (n=12), craving (n=5) and mood effect (n=2). Four studies examined moderators of treatment outcomes.
The first set of studies (Tables~\ref{tab:e1}, \ref{tab:e2}, \ref{tab:e3}) investigated alcohol use, craving and mood disturbances as clinical outcomes for ondansetron efficacy, while the second set (Table~\ref{tab:m}) examined the moderators of treatment outcomes. 
%%%%%%%%%%%%%%%%%%%%%%%%%%%%%%%%%%%%%%%%
\paragraph{Sample characteristics}

The included studies involved 11 distinct study populations whom characteristics are described in Table~\ref{tab:pop}.

In total, \totalrandomized\ subjects were enrolled, of which 1071 (98.4\%) met criteria for alcohol use disorder. Patients were mostly males (n=819; \pmales\%), with a mean age of \meanage.

Patients enrolled were diagnosed as alcohol dependent according to the DSM-III-TR (\cite{sellers_clinical_1994}, \cite{johnson_ondansetron_2000}), DSM-IV (\cite{myrick_effect_2008}, \cite{johnson_combining_2000}, \cite{ait-daoud_combining_2001}, \cite{kranzler_effects_2003}, \cite{dawes_prospective_2005}, \cite{johnson_pharmacogenetic_2011}), DSM-IV-TR (\cite{kenna_within-group_2009}, \cite{kenna_ondansetron_2014-1}), DSM-5 (\cite{sherwood_brown_randomized_2021}) or ICD-10 (\cite{correa_filho_pilot_2013}).
Some studies required additional criteria, such as more than 35 standard drinks per week for men or 28 for women (\cite{kenna_within-group_2009}, \cite{kenna_ondansetron_2014-1}), more than 30 drinks per week for men or 21 for women (\cite{seneviratne_serotonin_2012}), at least 15 standard drinks in the week before enrollement (\cite{sherwood_brown_randomized_2021}), more than 3 standard drinks per day and a Michigan Alcohol Screening Test greater than 5 (\cite{johnson_combining_2000}, \cite{johnson_ondansetron_2000}, \cite{ait-daoud_combining_2001}), an AUDIT score greater than 8 (\cite{johnson_pharmacogenetic_2011}) or a diagnose before the age of 25 (\cite{sherwood_brown_randomized_2021}).
Whereas most trials (\cite{sellers_clinical_1994}, \cite{johnson_combining_2000}, \cite{johnson_ondansetron_2000}, \cite{ait-daoud_combining_2001}, \cite{kranzler_effects_2003}, \cite{dawes_prospective_2005}) concerned treatment-seeking patients, some (\cite{myrick_effect_2008}, \cite{kenna_within-group_2009}, \cite{kenna_ondansetron_2014-1}) did not. With the exclusion of nicotine and alcohol, drug use and disorder was considered as an exclusion criteria in most included studies. Thus, participants who reported drug use (\cite{kranzler_effects_2003}, \cite{myrick_effect_2008}, \cite{kenna_within-group_2009}, \cite{johnson_pharmacogenetic_2011}, \cite{kenna_ondansetron_2014-1}) and/or had a positive drug screening test (\cite{sellers_clinical_1994}, \cite{johnson_combining_2000}, \cite{johnson_ondansetron_2000}, \cite{ait-daoud_combining_2001}, \cite{myrick_effect_2008}, \cite{correa_filho_pilot_2013}]) were often excluded, except for cannabis use in a few trials (\cite{dawes_prospective_2005}, \cite{myrick_effect_2008}). Receipt of alcohol use disorder treatment prior to enrollment was an exclusion criteria; one study (\cite{sellers_clinical_1994}) considered treatment over the previous 12 months, and 3 (\cite{johnson_age_2000}, \cite{ait-daoud_combining_2001}, \cite{dawes_prospective_2005}) over the previous 30 days. 

Some studies excluded participants who had been treated with stimulants, sedatives, hypnotics or with treatment that could have an effect on alcohol consumption or mood (\cite{correa_filho_pilot_2013}, \cite{johnson_ondansetron_2000}, \cite{johnson_combining_2000}, \cite{ait-daoud_combining_2001}, \cite{dawes_prospective_2005}). In one study (\cite{sherwood_brown_randomized_2021}), having been treated with naltrexone, acamprosate, disulfiram or topiramate 2 weeks prior inclusion, or current treatment with phenytoin, carbamazepine, rifampicine, apomorphine or tramodol (due to potential interactions with ondansetron) were exclusion criteria.

Psychiatric disorders were exclusion criteria in most studies (\cite{johnson_ondansetron_2000}, \cite{johnson_combining_2000}, \cite{ait-daoud_combining_2001} \cite{kranzler_effects_2003}, \cite{kenna_within-group_2009}, \cite{johnson_pharmacogenetic_2011}, \cite{kenna_ondansetron_2014}). Two studies considered only major diagnosis (\cite{myrick_effect_2008}) or clinically significant disorders (\cite{correa_filho_pilot_2013} \cite{dawes_prospective_2005}) as exclusion criteria. In the study of Sellers et al. (\cite{sellers_clinical_1994}), a Montgomery/asberg Depression scale score below 15 and a Spielberger State-Trait anxiety inventory score below 55 were required. In contrast, one study (\cite{sherwood_brown_randomized_2021}) enrolled only people with a concurrent psychiatric diagnosis.

Many studies also required good health at enrollment and notably excluded frequent AUD comorbidities such as elevated bilirubin (\cite{johnson_ondansetron_2000}, \cite{kenna_within-group_2009}), liver enzymes (\cite{dawes_prospective_2005}, \cite{kenna_within-group_2009}, \cite{sherwood_brown_randomized_2021}), liver cirrhosis (\cite{correa_filho_pilot_2013}, \cite{sherwood_brown_randomized_2021}) or severe alcohol withdrawal (\cite{johnson_combining_2000}, \cite{johnson_ondansetron_2000}, \cite{dawes_prospective_2005}, \cite{sherwood_brown_randomized_2021}).

%%%%%%%%
Considering participants subtypes, 9 studies (\cite{johnson_ondansetron_2000}, \cite{johnson_combining_2000}, \cite{ait-daoud_combining_2001}, \cite{ait-daoud_combining_2001-1}, \cite{johnson_ondansetron_2002}, \cite{johnson_ondansetron_2003}, \cite{kranzler_effects_2003}, \cite{roache_prediction_2008}), \cite{sherwood_brown_randomized_2021}) focused on clinical characteristics based on age of onset, before or after the age of 25 (EOA and LOA, while 6 studies examined patient subtypes according to their genotypes (\cite{johnson_pharmacogenetic_2011}, \cite{seneviratne_serotonin_2012}, \cite{johnson_determination_2013}, \cite{kenna_ondansetron_2014}, \cite{kenna_ondansetron_2014-1}, \cite{hou_subgroup_2015}). The most frequent investigated gene was SLC6A4, coding for the serotonin transporter that contains a polymorphism in the promoter region, the 5-HTT-linked polymorphic region, with a "short" (S) and "long" (L). Other genes of interest were HTR3A and HTR3B, which regulate the 5HT3 receptor had polymorphisms which influenced response to ondansetron (rs1150226-AG and rs1176713-GG in HTR3A and rs17614942-AC in HTR3B).

Ondansetron dosage was between 1 µg/kg bid (twice a day) and 16 mg per day with the most frequent dosage being 4 µg/kg bid. One study (\cite{sherwood_brown_randomized_2021}) used a flexible dosage which varied in function of treatment response and could range from 0.5 to 4 mg bid (with mean dose at exit being 3.24 ± 2.64 mg/day). Four studies (\cite{johnson_combining_2000}, \cite{ait-daoud_combining_2001}, \cite{ait-daoud_combining_2001-1}, \cite{myrick_effect_2008}) involving 127 patients in total used ondansetron in combination with naltrexone at a dosage of 50 mg per day. Three studies (97 patients) evaluated ondansetron against sertraline, at a dosage of  200 mg per day (\cite{kenna_within-group_2009}, \cite{kenna_ondansetron_2014}, \cite{kenna_ondansetron_2014-1}).
Treatment duration ranged from 8 days to 11 weeks. Pill count or riboflavin dosage were used for assessing treatment compliance

%%%%%%%%%%%%%%%%%%%%%%%%%%%%%%%%%%%%
%ICI !!! = page 6 du word

%%%%%%%%%%%%%%%%%%%%%%%%%%%%%%%%%%%%
\paragraph{Treatment outcomes} 
Efficacy was most often assessed by evaluating the number of standard drinks and derived variables such as defined by the Alcohol Timeline Followback (TLFB) method \cite{sobell1996timeline}. Drinking outcomes were drinks per day (DD), drinks per drinking day (DDD), percentage of day abstinent (PDA), heavy drinking days (days with more than 5 drinks per day) percentage of heavy drinking day (PHDD). Standard drink definition varied between different studies, it was defined as 12 g (\cite{johnson_ondansetron_2000}\cite{johnson_pharmacogenetic_2011}), 13 g (\cite{sellers_clinical_1994} or 14 g\cite{correa_filho_pilot_2013}) of pure ethanol.

Some studies (\cite{ait-daoud_combining_2001}, \cite{johnson_ondansetron_2002}\cite{myrick_effect_2008}) evaluated alcohol craving, either with a visual analogical scale or with the obsessive compulsive drinking scale (OCDS \cite{anton1996obsessive}). One study (\cite{dawes_reductions_2005}) used the Adolescent Obsessive–Compulsive Drinking Scale (A-OCDS) and another (\cite{sherwood_brown_randomized_2021}) the Penn Alcohol Craving Scale (PACS) to assess craving.

One study (\cite{myrick_effect_2008}) used functional magnetic resonance imaging to determine ventral striatum activation. Another study used the Profile of Mood States \cite{mcnair1989profile} to evaluate attenuation of mood disturbances.
Finally, one study (\cite{sherwood_brown_randomized_2021}) used Hamilton Rating Scale for Depression (HRSD), Young Mania Rating Scale (YMRS), and Inventory of Depressive Symptomatology–Self-report (IDS-SR).

A few studies measured carbohydrate deficient transferrin (CDT) (\cite{johnson_ondansetron_2000}, \cite{ait-daoud_combining_2001-1}, \cite{kranzler_effects_2003}, \cite{sherwood_brown_randomized_2021}) as an objective measure of alcohol use. Three studies (\cite{kenna_within-group_2009}, \cite{kenna_ondansetron_2014}, \cite{kenna_ondansetron_2014-1}) measured  or the volume of alcohol consumed in an alcohol self administration. $\gamma$-glutamyltransferase levels were used as an outcome in one study (\cite{sherwood_brown_randomized_2021}).
%variables/instruments for alcohol use and craving measures (cliniques/imagerie) 

%Impact of ondansetron on alcohol use and alcohol craving 
%%%%%%%%%%%%%%%%%%%%%%%%%%%%%%%%%%%%%

%%%%%%%%%%%%%%%%%%%%%%%%%%%%%%%%%%%%%%%%%%%%%%%%%%%%%%%%%%%%%%%%%%%%%%%%%%%%%%%%%%%%%%%%%%%%%%%%%%%%%%%%%%%%%%%%%%%%%%%%%%%%%%%%%%%%%%%

\paragraph{Study results}
The 21 clinical trials where subdivided in two main subgroups according to their primary outcomes. The first set of studies (Tables \ref{tab:e1}, \ref{tab:e2} and \ref{tab:e3}) evaluated the efficacy of ondansetron, by evaluating its impact on alcohol use, craving and mood disturbances. The second set (Table \ref{tab:m}) evaluated the moderators of treatment outcomes.
%%%%%%%%%%%%%%%%%%%%%%%%%%%%%%%%%%%%%%%%%%%%%%%%%%%%%%%%%%%%%%%%%%%%%%%%%%%%%%%%%%%%%%%%%%%%%%%%%%%%%%%%%%%%%%%%%%%%%%%%%%%%%%%%%%%%%%%

\paragraph{Alcohol use reduction}
Eleven studies, summarized in Table~\ref{tab:e1} evaluated the impact of ondansetron, alone or in combination with naltrexone on alcohol use. The main outcomes were mostly self-reported changes in alcohol consumption in standard drinks (using the previously defined TLFB method). A few studies used objective measures: plasma CDT (\cite{johnson_ondansetron_2000}, \cite{ait-daoud_combining_2001-1}, \cite{kranzler_effects_2003}, \cite{sherwood_brown_randomized_2021}), GGT (\cite{sherwood_brown_randomized_2021}) or the volume of alcohol consumed in an alcohol self administration  (\cite{kenna_within-group_2009}, \cite{kenna_ondansetron_2014}, \cite{kenna_ondansetron_2014-1}).

In a sample of 71 males suffering from alcohol dependence (DSM-III-TR), Sellers \textit{et al.} \cite{sellers_clinical_1994} showed a trend (p = 0,06), and \textit{post hoc} analysis indicated a significative impact on AUD patients drinking less than 10 drinks a day. The effect of ondansetron was shown to be non-linear, as 0,25 mg was more effective than 2 mg.

In a small scale randomized control trial \cite{johnson_combining_2000} (n = 20, exclusively EOA), a combination of ondansetron and naltrexone showed significant effect on reduction of drinks per day (0.99 $\pm$ 0.60 vs 3.68 $\pm$ 0.63, effect size = 1.42) and drinks per drinking day (3.14 $\pm$ 0.87 vs 6.76 $\pm$ 0.71, effect size = 1.71), as well as a trend in reducing the percentage of abstinent days, compared to placebo.

A subsequent analysis of the sample by Ait-Daoud \textit{et al.} showed that the combination of ondansetron and naltrexone was associated with significantly lower CDT levels \cite{ait-daoud_combining_2001-1}.

In a later trial (Johnson \textit{et al.} 2000b \cite{johnson_ondansetron_2000}), 271 patients of the 321 enrolled were given ondansetron at various dosages (1, 4 and 16 µg/kg of body weight, twice a day). In this study, ondansetron was found to be significantly more effective than placebo in reducing alcohol consumption among EOA but not LOA. Ondansetron at 4 µg/kg \emph{b.i.d.}, which was (non significantly) superior to the other dosages, was more effective than placebo on drinks per day (1.56 vs 3.30, p = 0.01), drinks per drinking day (4.28 vs 6.90, p = 0.004), percentage of day abstinent (70.10 vs 50.20, p = 0.02) and mean log CDT ratio (-0.19 vs 0.12, p = 0.01). Among EOA, all other dosages were superior to placebo on the two first criteria.

These results were subsequently replicated by Kranzler \textit{et al.} in 2003 \cite{kranzler_effects_2003}, who showed a significant reduction (compared to baseline) in most alcohol-related measures (drinks per day, drinks per drinking days, DrinC total score) among EOA and LOA who received ondansetron (4 µg/kg \emph{b.i.d.}. A significant difference was also found between EOA and LOA receiving ondansetron, benefiting the former on drinks per day, drinks per drinking day and DrinC total score. 

A larger-scale clinical trial, conducted by Johnson \textit{et al.} \cite{johnson_pharmacogenetic_2011}, enrolling 283 patients  showed that L/L-subjects receiving ondansetron significantly reduced their alcohol consumption, measured by drinks per drinking day and percentage of days abstinent as compared to placebo (respectively -1.62, p = 0.007 and 11.27\%, p = 0.023).

In a small-scale study with 15 non-treatment seeking individuals, Kenna \textit{et al.} \cite{kenna_within-group_2009} showed that patient with L/L genotype on the 5-HTTPLPR promoter region of SLC6A4 (further referred as L/L-subjects) that were administered ondansetron (4 µg/kg \emph{b.i.d.}) for 3 weeks drank significantly less alcohol at an alcohol self-administration, compared to similar patients administered sertaline (200 mg per day).

Another clinical trial involving 77 patients, showed limited support that ondansetron may reduce drinking in non-treatment seeking L/L-subjects and was inconclusive in evaluated the effectivness of sertraline in S/L or S/S-subjects (\cite{kenna_ondansetron_2014-1}). Further analysis (\cite{kenna_ondansetron_2014}) pinpointed gender differences as L/L women treated with ondansetron and S/L or S/S women treated with sertraline had significantly less drinks per drinking days and drank less at alcohol self-administration evaluations.

In the only trial taking place outside of North America, Corrêa Filho \textit{et al.} \cite{correa_filho_pilot_2013} showed a significative reduction of heavy drinking days (7,8 \% vs 11,7\%, p=0.02) but not of other measured outcomes.

Recently, Sherwood \textit{et al.} \cite{sherwood_brown_randomized_2021} evaluated the efficacy of ondansetron in 70 patients suffering from both amcohol use disorder and bipolar disorder. Results showed a trend in greater reduction of drinking as measured by the TLFB method that failed to reach significance.
%%%%%%%%%%%%%%%%%%%%%%%%%%%%%%%%%%%%%%%%%%%%%%%%%%%%%%%%%%%%%%%%%%%%%%%%%%%%%%%%%%%%%%%%%%%%%%%%%%%%%%%%%%%%%%%%%%%%%%%%%%%%%%%%%%%%%%%

\paragraph{Craving and cue-induced craving}
Four studies (Table~\ref{tab:e2}) evaluated the impact of ondansetron on craving.

An analysis of the study population of Johnson \textit{et al.} 2000a (\cite{johnson_combining_2000}) found that the combination of ondansetron and naltrexone was significantly better than placebo at reducing craving among EOA \cite{ait-daoud_combining_2001}.

Johnson \textit{et al.} 2002 \cite{johnson_ondansetron_2002} showed that ondansetron at 4 µg/kg \emph{b.i.d.} was associated with a significant reduction in craving (measured by visual analog scale) compared to the placebo group, but only in EOA. In contrast, craving was significantly increased by ondansetron at 1 µg/kg \emph{b.i.d.} among LOA.

In a BOLD-MRI laboratory study, Myrick \textit{et al.} \cite{myrick_effect_2008} evaluated ventral striatum activation of AUD suffering people, treated for 7 days by either naltrexone, ondansetron, a combination of both or placebo, and "social-drinkers" (control group), when shown alcohol cues or neutral beverage cues. Ventral striatum activation was significantly reduced in the naltrexone, combination and social-drinkers groups. This was correlated with reduced craving scores in these groups. Ondansetron alone wasn't significantly effective in the reduction of ventral striatum activation nor craving. There was however a trend in the reduction of both.

Sherwood \textit{et al.} \cite{sherwood_brown_randomized_2021} evaluated craving with the PACS but found no differences between ondansetron and placebo groups, among patients with bot bipolar disorder and AUD. 
%%%%%%%%%%%%%%%%%%%%%%%%%%%%%%%%%%%%%%%%%%%%%%%%%%%%%%%%%%%%%%%%%%%%%%%%%%%%%%%%%%%%%%%%%%%%%%%%%%%%%%%%%%%%%%%%%%%%%%%%%%%%%%%%%%%%%%%

\paragraph{Mood disturbances}
One study, presented in Table~\ref{tab:e3}, showed that among EAO only, ondansetron significantly reduced mood disturbances as measured by the POMS scale (\cite{mcnair1989profile}).

Sherwood \textit{et al.} \cite{sherwood_brown_randomized_2021} showed that ondansetron was significantly more efficacious than placebo in the reduction of HRSD scores, but not of YMRS or IDS-SR scores, among patients suffering from both bipolar disorder and AUD.
%%%%%%%%%%%%%%%%%%%%%%%%%%%%%%%%%%%%%%%%%%%%%%%%%%%%%%%%%%%%%%%%%%%%%%%%%%%%%%%%%%%%%%%%%%%%%%%%%%%%%%%%%%%%%%%%%%%%%%%%%%%%%%%%%%%%%%%

\paragraph{Moderators of treatment outcomes}
Finally, six studies, presented in Table~\ref{tab:m} didn't directly evaluate the efficacy of ondansetron but provided useful information on predicton factors or safety and tolerability.

As ondansetron had previously been found useful in early onset alcoholism, Dawes \textit{et al.} \cite{dawes_prospective_2005} evaluated ondansetron among adolescent in a prospective, open-label trial, which showed that ondansetron was safe and well tolerated in adolescents with alcohol dependence. In a subsequent study \cite{dawes_reductions_2005}, they found that reduction of drinking (as assest by TLFB) was correlated with reduction in craving, as measured by POCS.

Roache \textit{et al.} 2008 \cite{roache_prediction_2008} compared the prediction capabilities of the EOA/LOA typology to the type A/type B typology precedently described by Babor \textit{et al.} 1992 \cite{babor1992a}\cite{babor1992b}(derived from Type I/II description by Cloninger \textit{et al.} \cite{cloninger1987systematic}). The A/B typology better described baseline severity of alcohol dependence but treatment response to ondansetron was significantly better predicted by the EOA/LOA typology.

Seneviratne \textit{et al.} \cite{seneviratne_serotonin_2012} produced some evidence that 5'-HTTPLR mRNA levels could be used as biomarker to evaluate treatment effectiveness in L/L-subjects treated with ondansetron.

Two studies analysed the population of Johnson \textit{et al.} 2011 \cite{johnson_pharmacogenetic_2011} and identified genotypes predicting treatment success. Johnson \textit{et al.} 2013 \cite{johnson_determination_2013} found 5 genotypes which presence predicted efficacy of ondansetron and which where present in a third of the population. Finally,  Hou \textit{et al.} \cite{hou_subgroup_2015} further worked on identifying ways of predicting effectivness of ondansetron and found that data mining approaches, such as interaction trees and virtual twins could simplify subgroup identification while limiting statistical errors.
%%%%%%%%%%%%%%%%%%%%%%%%%%%%%%%%%%%%%%%%%%%%%%%%%%%%%%%%%%%%%%%%%%%%%%%%%%%%%%%%%%%%%%%%%%%%%%%%%%%%%%%%%%%%%%%%%%%%%%%%%%%%%%%%%%%%%%%%%%%%%%%%
