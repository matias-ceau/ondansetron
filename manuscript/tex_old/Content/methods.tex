\paragraph{Research design}
The study involved a systematic review of the literature based on the Preferred Reporting Items for Systematic reviews and Meta-Analyses (PRISMA) guidelines \cite{page2021prisma}.

\paragraph{Databases and search strategy}
This review was based on the following databases: PUBMED/MEDLINE, Psychinfo, Cochrane, Wiley Online Library. The search was performed for all years up to December, 2021. 

The following search terms were used:

For Medline search, the relevant articles were identified by combining the terms: ("ondansetron"[MeSH Terms] OR "ondansetron"[All Fields]) AND ("alcoholism"[MeSH Terms] OR "alcoholism"[All Fields] OR "alcohol use disorder"[All Fields] OR "alcohol abuse"[All Fields] OR "AUD"[All Fields])

For the Wiley Online Library search we used the keywords "ondansetron" AND "alcoholism".

For the PsycInfo search, the keywords were "ondansetron or Zofran" AND "alcoholism or alcohol dependence or alcohol abuse or alcoholic or alcohol addiction".

Finally, the Cochrane Library was used by searching ("alcohol use disorder" OR "alcohol dependance") AND "ondansetron" in "Title Abstract Keyword".

\paragraph{Eligibility criteria}
Studies were included if they met the following inclusion criteria :
\begin{itemize}
 \item Reported as a peer reviewed journal
    \item Concerning individuals suffering from AUD, with no restrictive criteria regarding age, sex, ethnic origin, or place of living.
    \item Assessing the impact of ondansetron on AUD and/or predictors of ondansetron treatment response.
    \item Papers published in English.
\end{itemize}

Studies were excluded if: 
\begin{itemize}
    \item Reviews, opinion papers, protocols, case reports,
    \item Preclinical studies
    \item Studies in healthy volunteers
    \item Not published in English
\end{itemize}

\paragraph{Study selection}
Two authors independently examined all titles and abstracts. Relevant articles were obtained in full-text and assessed for inclusion criteria separately by the two reviewers based on the inclusion and exclusion criteria previously mentioned. Disagreements were resolved via discussion of each article for which conformity to inclusion and exclusion criteria were uncertain and a consensus was reached. The reference lists of major papers were also manually screened in order to ensure comprehensiveness of the review. All selected studies were read in full to confirm inclusion criteria, study type and study population.

\paragraph{Assessment of risk of bias in included studies} 
Two review authors independently assessed the risk of bias of each included study using the revised Cochrane tool for assessing risk of bias in randomised trials (RoB 2 \cite{sterne2019rob}), in accordance with methods recommended by Cochrane collaboration. The risk-of-bias plot in Figure~\ref{fig:trafficlight} was generated using the Robvis online tool \cite{mcguinness_risk--bias_2021}. The following judgements were used : high risk, low risk or unclear (either lack of information or uncertainty over the potential for bias). Authors resolved disagreements by consensus, and a third author was consulted to resolve disagreements if necessary.

The Newcastle Ottawa Scale (NOS, \cite{peterson2011newcastle}) was used for assessing single-arm non randomized studies. However, it had to be adapted by removing the Comparability item for two of the studies that lacked a control group.

\paragraph{Data collection}
Sample characteristics (including socio-demographic data, age, gender, comorbidity), information on study design, eligibility criteria and exclusion criteria, study duration and location, clinical outcome assessment methods, dropout rate and treatment dosage were extracted.