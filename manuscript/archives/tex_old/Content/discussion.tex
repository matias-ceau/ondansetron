To our knowledge, this paper is the only systematic review including every clinical trial involving the use of ondansetron for the treatment of alcohol use disorder. A subsequent research identified 5 systematic reviews in the last 10 years that included the keywords 'ondansetron' and 'alcohol use disorder' or 'alcoholism'. Bauer \textit{et al.} 2015 \cite{bauer_serotonergic_2015} focused on the influence of serotonergic gene variation in substance use pharmacotherapy and included four out of the 21 studies presented here. Naglich \textit{et al.} 2018 \cite{naglich_systematic_2018} focused on combined pharmacotherapy for the treament of alcohol use and thus included 2 studies involving ondansetron and naltrexone. Cservenka \textit{et al.} 2017 \cite{cservenka_review_2017} focused on pharmacogenetics and the implication of ethnic diversity in the treatment of AUD and included 2 papers. Castrén \textit{et al.} 2019 \cite{castren_selecting_2019} focused on the recent findings in AUD pharmacoterapy and mentioned ondansetron without including any of the clinical trials. Finally, Bharadwaj \textit{et al.} 2018 \cite{bharadwaj_pharmacotherapy_2018} focused on the pharmacotherapy for relapse prevention in AUD in the Indian setting and also mentioned ondansetron but didn't include any trial on this topic.

One review \cite{thompson_variation_2016} focuses on the role of the serotonin transporter gene in AUD and thus cites 6 of the latest pharmacogenetics studies.

Most of the clinical trials described in this systematic review have stringent inclusion criteria which greatly limits their external validity. Particularly, patients suffering from dual diagnosis or addicted to several substance (with the exception of nicotine) were often excluded. 

The high dropout rate (mean dropout rate 35.4\%) could impact the validity of the findings, but evidence to identify whether or not dropout rate favors medication is lacking.

Finally, out of the seven registered trials that have no published papers yet, three haven't had any updates for more than ten years whereas one reported non significant results. This may pose a publication bias that is to be taken in consideration.

\paragraph{Conclusion} Whereas growing evidence tends to suggest efficacy of ondansetron as a treatment of alcohol use disorder in particular genetic subgroups, nsew studies will be needed to fully conclude. In particular, there is a need for bigger studies evaluating long term changes in alcohol consumption. These studies should also have less exclusion criteria to maximize their external validity.

%%%%%%%%%%%%%%%%%%%%%%%%%%%%%%%%%%%%%%%%%%%%%%%%%%%%%%%%%%%%%%%%%
%10 dernière années :

%- la revue la plus complète contient 4 des articles et évalue son efficacité (mais se concentre plus généralement sur tous les médicaments qui influence la sérotonine) (Bauer 2015)

%- une revue inlcut 2 des articles, et porte sur les combinaison de traitement dans le TUA 'naglich)

%- une revue inclut 2 des articles, mais porte sur les conséquences de l'ethnicité sur la réponse aux traitements (Cservenka 2017)

%- une revue l'aborde mais n'inclut aucune des études (sujet : pharmacothérapie du TUA, sur les 2 dernières années) (Castren 2019)

%- la dernière mentionne l'ondansetron comme potentiellement intéressant mais déclare qu'il n'y a pas d'études disponibles sur le sujet (Bharadwaj 2018).

%%%%%%%%%%%%%%%%%%%%%%%%%%%%%%%%%%%%%%%%%%%%%%%%%%%%%%%%%%%%%%%%%

%une revue non systématique porte sur un sujet très proche, les variations du gène de la sérotonine et le TUA, et mentionne les 6 études de pharmacogénétiques (l'article est écrit par les auteurs de ces études) (Thomson 2016)