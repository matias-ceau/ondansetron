%%%%%%%%%%%%%%%%%%%%%%%%%%%%
\paragraph{}
have shown differential effects among alcohol-use disorder suffering patients. Different typologies have been previously proposed, based on personality such as type I/II (Cloninger et al. 1987 \cite{cloninger1987systematic}) or type A/B personalities (Babor et al. 1992 \cite{babor1992a}, \cite{babor1992b}) or simply defined by the age of onset of alcoholism (Early-onset alcoholism/Late-onset alcoholism (EOA/LOA),  ]) caracterized by differential course, prognosis and treatment responses. Results from these studies indicated distinct effect of ondansetron among EOA compared to LOA, which is hypothesized to be linked to genetic variations among individuals.
\paragraph{}
As the functional state of the serotonin transporter protein (5-HTT) is an important factor of the serotonergic function control, more recent pharmacogenetic studies have investigated the potential role of 5-HTT genotype on drinking behaviors and alcohol craving (\cite{ait2009can}). 5-HTT gene polymorphisms, involving two variants, a short form (S) and a long form (L), have been shown to be associated with differential 5-HT neurotransmission, which could moderate the rewarding effects and the craving for alcohol, and thereby ondansetron treatment response among AUD suffering patients.

\section{REVISION 27/01/22}
\subsection{INTRO}
\paragraph{}
As the functional state of the serotonin transporter protein (5-HTT) is an important factor of the serotonergic function control, more recent pharmacogenetic studies have investigated the potential role of 5-HTT genotype on drinking behaviors and alcohol craving (\cite{ait2009can}). 5-HTT gene polymorphisms, involving two variants, a short form (S) and a long form (L), have been shown to be associated with differential 5-HT neurotransmission, which could moderate the rewarding effects and the craving for alcohol, and thereby ondansetron treatment response among AUD suffering patients.

Direct inhibition of the $\text{5-HT}_3$ receptor is thus hypothesized to alleviate alcohol craving, that is currently considered as a key determinant of relapse vulnerability as well as a major treatment target. A better knowledge of the potential impact of 5-HT antagonist medication on alcohol consumption and craving in AUD, as well as individual caracteristics and genotype associated with treatment response is therefore a critical issue to improve treatment approaches and develop personalized medicine in the pharmacotherapy of alcohol use disorder.
\paragraph{}
We conducted a systematic review of the literature in order to summarize scientific evidence of the efficacy of ondansetron on alcohol consumption and craving as well as clinical or genetic predictors of treatment response.

\subsection{METHODS}
\paragraph{}
A PubMed search was conducted in the MEDLINE database, the Wiley Online Library, PsycInfo database and.
%COCHRANE

For Medline search, the relevant articles were identified by combining the terms:
("ondansetron"[MeSH Terms] OR "ondansetron"[All Fields]) AND ("alcoholism"[MeSH Terms] OR "alcoholism"[All Fields] OR "alcohol use disorder"[All Fields] OR "alcohol abuse"[All Fields] OR "AUD"[All Fields])

For the Wiley Online Library search we used the keywords "ondansetron" AND "alcoholism".

For the PsycInfo search, the keywords were "ondansetron or zofran" AND "alcoholism or alcohol dependence or alcohol abuse or alcoholic or alcohol addiction".

Finally, the Cochrane Library was used by searching ("alcohol use disorder" OR "alcohol dependance") AND "ondansetron" in "Title Abstract Keyword".

Two authors independently examined titles and abstracts. Relevant articles were obtained in full-text and assessed for inclusion criteria blindly by the two reviewers. Disagreement was resolved via discussion to reach consensus. Data from the eligible articles were independently extracted by two reviewers using a standardized data extraction form. Extracted data included participant characteristics, study design, treatment outcomes and results, how confounders were controlled for, and limitations.  
