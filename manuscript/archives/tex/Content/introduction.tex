\paragraph{}

Alcohol use disorder (AUD) is a heterogeneous and chronic relapsing disorder resulting in a complex interaction between neurobiological, genetic and environmental factors. Despite the demonstrated efficacy of some approved medications (acamprosate, naltrexone, disulfiram), a key barrier is the fact that these medications are not effective in every patient pointing out the need for more personalized therapy approaches to overcome this heterogeneity. In this perspective, major advances in pharmacogenetics have highlighted distinct clinical subgroups of AUD according to genetic variation, that could be associated with differential treatment responses. Thus, the identification of patient subtypes that are most likely to respond favorably to different medications is crucial with a need for a better targeting of medication to specific patients.

\paragraph{}

Amongst the emerging pharmacotherapies for AUD, Ondansetron (IUPAC name: (RS)-9-Methyl-3- [(2-methyl-1H-imidazol1-yl)methyl]-2,3-dihydro-1Hcarbazol-4(9H)-one), a selective antagonist of the $\text{5-HT}_3$ receptor, has shown some promising results. Ondansetron is approved by ANSM in France and FDA in the USA as an antiemetic for cancer treatment-induced and anesthesia related nausea and vomiting. 
In the late 1980s, Hagan et al. \cite{hagan_effect_1987} showed that the injection of ondansetron in the ventral tegmental area of the rat brain lessened induced hyperactivation in the nucleus accumbens, pointing out the tight relationship between serotonin function and the mesolimbic dopaminergic reward system. Based on these findings and the role of dopaminergic activity on the rewarding effects of alcohol, ondansetron was thought to attenuate the pleasurable subjective effects of alcohol and thereby to reduce alcohol consumption in AUD suffering patients. Some phase 1 clinical studies (\cite{grant_blockade_1991}, \cite{johnson_attenuation_1993}) found promising results in healthy male volunteers, and provided preliminary evidence on the role of ondansetron in reducing the reinforcing properties of alcohol and the desire to use by 5-HT3 receptor blockade. 
Later phase 2 clinical studies (\cite{johnson_ondansetron_2000}, \cite{kranzler_effects_2003}) suggested differential effects among AUD patients depending on the age of onset (Early-onset alcoholism EOA/Late-onset alcoholism LOA) \cite{varma1994correlates}, which is hypothesized to be linked to individual genetic variations. Interestingly, compared to placebo, ondansetron was associated with reduced drinking and significant reduction in overall craving in randomized placebo-controlled studies (\cite{johnson_ondansetron_2000}, \cite{johnson_ondansetron_2002}), but only among patients who developed AUD before age 25 only (EOA), and not among late-onset patients (LOA), presumably by ameliorating serotonergic abnormality.

\paragraph{}Furthermore, as the functional state of the serotonin transporter protein (5-HTT) is an important factor of the serotonergic function control, more recent pharmacogenetic studies have investigated the potential role of 5-HTT genotype on drinking behaviors and alcohol craving ([7]). 5-HTT gene polymorphisms, involving two variants, a short form (S) and a long form (L), have been shown to be associated with differential 5-HT neurotransmission, which could moderate the rewarding effects and the craving for alcohol, and thereby ondansetron treatment response among AUD suffering patients. In line with this hypothesis, ondansetron was administered in a large study among AUD patients according to the 5-HTT polymorphism (ref Johnson 2011). Participants with the LL genotype significantly reduced their drinking compared to the LS or SS genotype, suggesting that ondansetron could represent an interesting approach for the personalized treatment of AUD according to specific polymorphism of the 5-HTT gene.

\paragraph{}Direct inhibition of the 5-HT3 receptor is thus hypothesized to reduce alcohol use and alleviate alcohol craving, that is currently considered as a key determinant of relapse vulnerability as well as a major treatment target. A better knowledge of the potential impact of 5-HT antagonist medication on alcohol consumption and craving in AUD, as well as individual clinical subtypes and genotype associated with treatment response is therefore a critical issue to improve treatment approaches and develop personalized medicine in the pharmacotherapy of alcohol use disorder. The aim of this systematic review is to address this issue by assessing scientific evidence of the efficacy of ondansetron on alcohol consumption and craving as well as clinical or genetic predictors of treatment response.
