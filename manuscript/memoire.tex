% Options for packages loaded elsewhere
\PassOptionsToPackage{unicode}{hyperref}
\PassOptionsToPackage{hyphens}{url}
%
\documentclass[
  12pt,
]{article}
\usepackage{amsmath,amssymb}
\usepackage{iftex}
\ifPDFTeX
  \usepackage[T1]{fontenc}
  \usepackage[utf8]{inputenc}
  \usepackage{textcomp} % provide euro and other symbols
\else % if luatex or xetex
  \usepackage{unicode-math} % this also loads fontspec
  \defaultfontfeatures{Scale=MatchLowercase}
  \defaultfontfeatures[\rmfamily]{Ligatures=TeX,Scale=1}
\fi
\usepackage{lmodern}
\ifPDFTeX\else
  % xetex/luatex font selection
\fi
% Use upquote if available, for straight quotes in verbatim environments
\IfFileExists{upquote.sty}{\usepackage{upquote}}{}
\IfFileExists{microtype.sty}{% use microtype if available
  \usepackage[]{microtype}
  \UseMicrotypeSet[protrusion]{basicmath} % disable protrusion for tt fonts
}{}
\makeatletter
\@ifundefined{KOMAClassName}{% if non-KOMA class
  \IfFileExists{parskip.sty}{%
    \usepackage{parskip}
  }{% else
    \setlength{\parindent}{0pt}
    \setlength{\parskip}{6pt plus 2pt minus 1pt}}
}{% if KOMA class
  \KOMAoptions{parskip=half}}
\makeatother
\usepackage{xcolor}
\usepackage[margin=30mm]{geometry}
\setlength{\emergencystretch}{3em} % prevent overfull lines
\providecommand{\tightlist}{%
  \setlength{\itemsep}{0pt}\setlength{\parskip}{0pt}}
\setcounter{secnumdepth}{-\maxdimen} % remove section numbering
% definitions for citeproc citations
\NewDocumentCommand\citeproctext{}{}
\NewDocumentCommand\citeproc{mm}{%
  \begingroup\def\citeproctext{#2}\cite{#1}\endgroup}
\makeatletter
 % allow citations to break across lines
 \let\@cite@ofmt\@firstofone
 % avoid brackets around text for \cite:
 \def\@biblabel#1{}
 \def\@cite#1#2{{#1\if@tempswa , #2\fi}}
\makeatother
\newlength{\cslhangindent}
\setlength{\cslhangindent}{1.5em}
\newlength{\csllabelwidth}
\setlength{\csllabelwidth}{3em}
\newenvironment{CSLReferences}[2] % #1 hanging-indent, #2 entry-spacing
 {\begin{list}{}{%
  \setlength{\itemindent}{0pt}
  \setlength{\leftmargin}{0pt}
  \setlength{\parsep}{0pt}
  % turn on hanging indent if param 1 is 1
  \ifodd #1
   \setlength{\leftmargin}{\cslhangindent}
   \setlength{\itemindent}{-1\cslhangindent}
  \fi
  % set entry spacing
  \setlength{\itemsep}{#2\baselineskip}}}
 {\end{list}}
\usepackage{calc}
\newcommand{\CSLBlock}[1]{\hfill\break\parbox[t]{\linewidth}{\strut\ignorespaces#1\strut}}
\newcommand{\CSLLeftMargin}[1]{\parbox[t]{\csllabelwidth}{\strut#1\strut}}
\newcommand{\CSLRightInline}[1]{\parbox[t]{\linewidth - \csllabelwidth}{\strut#1\strut}}
\newcommand{\CSLIndent}[1]{\hspace{\cslhangindent}#1}
\ifLuaTeX
  \usepackage{selnolig}  % disable illegal ligatures
\fi
\IfFileExists{bookmark.sty}{\usepackage{bookmark}}{\usepackage{hyperref}}
\IfFileExists{xurl.sty}{\usepackage{xurl}}{} % add URL line breaks if available
\urlstyle{same}
\hypersetup{
  hidelinks,
  pdfcreator={LaTeX via pandoc}}

\title{\textbf{Ondansetron in alcohol use disorder:}}
\usepackage{etoolbox}
\makeatletter
\providecommand{\subtitle}[1]{% add subtitle to \maketitle
  \apptocmd{\@title}{\par {\large #1 \par}}{}{}
}
\makeatother
\subtitle{a systematic review}
\author{Matias Ceau\\
Pr Mélina Fatseas}
\date{November 2024}

\begin{document}
\maketitle
\begin{abstract}
\textbf{Context} Alcohol use disorder (AUD) is a frequent disorder. Few
treatments have shown a great efficacy in clinical studies. It is
hypothetized that this is due to the fact that AUD suffering people
constitute a heterogeneous group composed of various endophenotypes.
Ondansetron, a selective 5HT3 receptor antagonist, has been evaluated in
AUD and partciularly in certain subgroups of patients. This review
presents clinical studies evaluating its impact.\\
\textbf{Methods} A systematic review was conducted on Pubmed and Wiley
Online Library. All 19 clinical trials involving the use of ondansetron
in AUD where included.\\
\textbf{Results} Ondansetron was found to be effective mostly in certain
subgroups of AUD-suffering patients. The identified subgroups were based
on clinical evaluation (age of onset, personality type) and genotype. A
number of limitations remains and further studies are needed.\\
\textbf{Key words} alcohol use disorder, ondansetron, 5-HT3 receptors,
serotonin, craving, pharmacogenetics, personnalized medicine*
\end{abstract}

{
\setcounter{tocdepth}{3}
\tableofcontents
}
\section{Introduction}\label{introduction}

Alcohol use disorder (AUD) is a heterogeneous and chronic relapsing
disorder resulting in a complex interaction between neurobiological,
genetic and environmental factors. Despite the demonstrated efficacy of
some approved medications (acamprosate, naltrexone, disulfiram), a key
barrier is the fact that these medications are not effective in every
patient pointing out the need for more personalized therapy approaches
to overcome this heterogeneity. In this perspective, major advances in
pharmacogenetics have highlighted distinct clinical subgroups of AUD
according to genetic variation, that could be associated with
differential treatment responses. Thus, the identification of patient
subtypes that are most likely to respond favorably to different
medications is crucial with a need for a better targeting of medication
to specific patients.

Amongst the emerging pharmacotherapies for AUD, Ondansetron (IUPAC name:
(RS)-9-Methyl-3-
{[}(2-methyl-1H-imidazol1-yl)methyl{]}-2,3-dihydro-1Hcarbazol-4(9H)-one),
a selective antagonist of the 5-HT3 receptor, has shown some promising
results. Ondansetron is approved by ANSM in France and FDA in the USA as
an antiemetic for cancer treatment-induced and anesthesia related nausea
and vomiting. In the late 1980s, Hagan \emph{et al.} (Hagan et al. 1987)
showed that the injection of ondansetron in the ventral tegmental area
of the rat brain lessened induced hyperactivation in the nucleus
accumbens, pointing out the tight relationship between serotonin
function and the mesolimbic dopaminergic reward system. Based on these
findings and the role of dopaminergic activity on the rewarding effects
of alcohol, ondansetron was thought to attenuate the pleasurable
subjective effects of alcohol and thereby to reduce alcohol consumption
in AUD suffering patients. Some phase 1 clinical studies ((Grant and
Barrett 1991), (B. A. Johnson et al. 1993)) found promising results in
healthy male volunteers, and provided preliminary evidence on the role
of ondansetron in reducing the reinforcing properties of alcohol and the
desire to use by 5-HT3 receptor blockade. Later phase 2 clinical studies
((Bankole A. Johnson, Roache, et al. 2000) , (Kranzler et al. 2003))
suggested differential effects among AUD patients depending on the age
of onset (Early-onset alcoholism EOA/Late-onset alcoholism LOA) (Varma
et al. 1994), which is hypothesized to be linked to individual genetic
variations. Interestingly, compared to placebo, ondansetron was
associated with reduced drinking and significant reduction in overall
craving in randomized placebo-controlled studies ((Bankole A. Johnson,
Roache, et al. 2000) + Johnson 2002), but only among patients who
developed AUD before age 25 only (EOA), and not among late-onset
patients (LOA), presumably by ameliorating serotonergic abnormality

Furthermore, as the functional state of the serotonin transporter
protein (5-HTT) is an important factor of the serotonergic function
control, more recent pharmacogenetic studies have investigated the
potential role of 5-HTT genotype on drinking behaviors and alcohol
craving ((Ait-Daoud et al. 2009)). 5-HTT gene polymorphisms, involving
two variants, a short form (S) and a long form (L), have been shown to
be associated with differential serotonin neurotransmission, which could
moderate the rewarding effects and the craving for alcohol, and thereby
ondansetron treatment response among AUD suffering patients. In line
with this hypothesis, ondansetron was administered in a large study
among AUD patients according to the 5-HTT polymorphism {[}ref Johnson
2011{]}. Participants with the LL genotype significantly reduced their
drinking compared to the LS or SS genotype, suggesting that ondansetron
could represent an interesting approach for the personalized treatment
of AUD according to specific polymorphism of the 5-HTT gene.

Direct inhibition of the 5HT3 receptor is thus hypothesized to reduce
alcohol use and alleviate alcohol craving, that is currently considered
as a key determinant of relapse vulnerability as well as a major
treatment target. A better knowledge of the potential impact of 5-HT
antagonist medication on alcohol consumption and craving in AUD, as well
as individual clinical subtypes and genotype associated with treatment
response is therefore a critical issue to improve treatment approaches
and develop personalized medicine in the pharmacotherapy of alcohol use
disorder. The aim of this systematic review is to address this issue by
assessing scientific evidence of the efficacy of ondansetron on alcohol
consumption and craving as well as clinical or genetic predictors of
treatment response.

\section{Methods}\label{methods}

\subsection{2.1 Research design}\label{research-design}

The study involved a systematic review of the literature based on the
Preferred Reporting Items for Systematic reviews and Meta-Analyses
(PRISMA) guidelines {[}35{]}).

\subsection{2.2. Databases and search
strategy}\label{databases-and-search-strategy}

This review was based on the following databases: PUBMED/MEDLINE,
Psychinfo, Cochrane, Wiley Online Library. The search was performed for
all years up to November, 2024.

The following search terms were used:

For Medline search, the relevant articles were identified by combining
the terms:

\begin{verbatim}
("ondansetron"[MeSH Terms] OR "ondansetron"[All Fields]) AND ("alcoholism"[MeSH Terms] OR "alcoholism"[All Fields] OR "alcohol use disorder"[All Fields] OR "alcohol abuse"[All Fields] OR "AUD"[All Fields])
\end{verbatim}

For the Wiley Online Library search we used the keywords
\texttt{"ondansetron"\ AND\ "alcoholism"}.

For the PsycInfo search, the keywords were
\texttt{"ondansetron\ or\ zofran"\ AND\ "alcoholism\ or\ alcohol\ dependence\ or\ alcohol\ abuse\ or\ alcoholic\ or\ alcohol\ addiction"}.

Finally, the Cochrane Library was used by searching
\texttt{("alcohol\ use\ disorder"\ OR\ "alcohol\ dependance")\ AND\ "ondansetron"\ in\ "Title\ Abstract\ Keyword"}.

\subsection{2.3. Eligibility criteria}\label{eligibility-criteria}

Studies were included if they met the following inclusion criteria : +
Reported as a peer reviewed journal + Concerning individuals suffering
from AUD, with no restrictive criteria regarding age, sex, ethnic
origin, or place of living. + Assessing the impact of ondansetron on AUD
and/or predictors of ondansetron treatment response + Papers published
in English

Studies were excluded if they were : - Preclinical studies, reviews,
opinion papers, protocols, case reports, - Animal studies - Studies in
healthy volunteers - Not published in English

\subsection{2.4. Study selection}\label{study-selection}

Two authors independently examined all titles and abstracts. Relevant
articles were obtained in full-text and assessed for inclusion criteria
separately by the two reviewers based on the inclusion and exclusion
criteria previously mentioned. Disagreements were resolved via
discussion of each article for which conformity to inclusion and
exclusion criteria were uncertain and a consensus was reached. The
reference lists of major papers were also manually screened in order to
ensure comprehensiveness of the review. All selected studies were read
in full to confirm inclusion criteria, study type and study population.

\subsection{2.5. Assessment of risk of bias in included
studies}\label{assessment-of-risk-of-bias-in-included-studies}

Two review authors independently assessed the risk of bias of each
included study using the revised Cochrane tool for assessing risk of
bias in randomized trials (RoB 2 (Sterne et al. 2019)), in accordance
with methods recommended by Cochrane collaboration. The risk-of-bias
plot in Figure 2 was generated using the Robvis online tool (McGuinness
and Higgins 2020). The following judgments were used : high risk, low
risk or unclear (either lack of information or uncertainty over the
potential for bias). Authors resolved disagreements by consensus, and a
third author was consulted to resolve disagreements if necessary.

The Newcastle Ottawa Scale (NOS, (Peterson et al. 2011)) was used for
assessing single-arm non-randomized studies. However, it had to be
adapted by removing the Comparability item for two of the studies that
lacked a control group.

\subsection{2.6. Collecting data}\label{collecting-data}

\section{Results}\label{results}

\subsection{3.1. Study selection}\label{study-selection-1}

A total of 90 results were found in the MEDLINE database, 68 on
PsycInfo, 35 on Cochrane Library and 8 additional studies were found on
Wiley Online Library. A total of 134 articles were identified through
the search of the databases. After reading the full text, 21 met
inclusion criteria for this review. This process is described in the
PRISMA flowchart \textbf{(Figure 1).} Among the 21 included studies, 18
were randomized controlled trials (RCT) or analysis of previous RCT, and
3 were prospective open-label studies ((Kranzler et al. 2003), (Dawes,
Johnson, Ait-Daoud, et al. 2005), (Dawes, Johnson, Ma, et al. 2005)).
The study duration ranged from 2 to 12 weeks. The selected articles were
published between 1994 and 2015.

\subsection{3.2. Quality and risk of bias
assessment}\label{quality-and-risk-of-bias-assessment}

Randomized controlled trials where analyzed using the Cochrane Rob2 tool
(see Figure 2 for the traffic-light plot). Two studies had a low concern
of bias ((Sellers et al. 1994), (Myrick et al. 2008)), most of the
studies showed some concern of bias and one had a high concern ((Corrêa
Filho and Baltieri 2013)). The high dropout rate of the studies was the
most concerning factor and affected the Domain 3 of the Risk of Bias
tool which represent bias due to missing outcome result. For the 3
prospective study, the Newcastle-Ottawa scale was used (see Figure 6).
The risk of bias was evaluated as acceptable as one study scored 7 out
of 9 possible points and the two others scored 6 out of 7 (adapted
score).

\subsection{3.3. Study results}\label{study-results}

Results are presented according to their primary outcomes. Among the 21
included studies, x assessed odansetron efficacy through alcohol use
(n=11), craving (n=) and mood effect (n=). X studies examined moderators
of treatment outcomes. The first set of studies \textbf{(Tables 2, 3 and
4)} investigated alcohol use, craving and mood disturbances as clinical
outcomes for odansetron efficacy, while the second set (\textbf{Table
5}) examined the moderators of treatment outcomes.

\subsubsection{3.3.1 Sample
characteristics}\label{sample-characteristics}

The included studies involved 11 distinct study populations whom
characteristics are described in \textbf{Table 1.}

In total, !1088 subjects were enrolled, of which !1071 !(\%) met
criteria for alcohol use disorder. Patients were mostly males !n= ;!75.3
\% with a mean age of !41.0 years.

Patients enrolled were diagnosed as alcohol dependent according to the
DSM-III-TR ((Sellers et al. 1994), (Bankole A. Johnson, Roache, et al.
2000)), DSM-IV ((Myrick et al. 2008), (Bankole A. Johnson, Ait‐Daoud,
and Prihoda 2000), (Ait-Daoud et al. 2001), (Kranzler et al. 2003),
(Dawes, Johnson, Ait-Daoud, et al. 2005), (Bankole A. Johnson et al.
2011)), DSM-IV-TR ((Kenna et al. 2009), (Kenna, Zywiak, Swift, McGeary,
Clifford, Shoaff, Vuittonet, et al. 2014)), DSM-5 ((Brown et al. 2021))
or ICD-10 ((Corrêa Filho and Baltieri 2013)). Some studies required
additional criteria, such as more than 35 standard drinks per week for
men or 28 for women ((Kenna et al. 2009), (Kenna, Zywiak, Swift,
McGeary, Clifford, Shoaff, Vuittonet, et al. 2014)), more than 30 drinks
per week for men or 21 for women ((Seneviratne and Johnson 2012)), at
least 15 standard drinks in the week before enrollement ((Brown et al.
2021)), more than 3 standard drinks per day and a Michigan Alcohol
Screening Test greater than 5 ((Bankole A. Johnson, Ait‐Daoud, and
Prihoda 2000), (Bankole A. Johnson, Roache, et al. 2000), (Ait-Daoud et
al. 2001)), an AUDIT score greater than 8 ((Bankole A. Johnson et al.
2011)) or a diagnosis before the age of 25 ((Brown et al. 2021)).
Whereas most trials ((Sellers et al. 1994), (Bankole A. Johnson,
Ait‐Daoud, and Prihoda 2000), (Bankole A. Johnson, Roache, et al. 2000),
(Ait-Daoud et al. 2001), (Kranzler et al. 2003), (Dawes, Johnson,
Ait-Daoud, et al. 2005)) concerned treatment-seeking patients, some
((Myrick et al. 2008), (Kenna et al. 2009), (Kenna, Zywiak, Swift,
McGeary, Clifford, Shoaff, Vuittonet, et al. 2014)) did not. Drug use
and disorder was considered as an exclusion criteria in most included
studies. Thus, participants who reported drug use ((Kranzler et al.
2003), (Myrick et al. 2008), (Kenna et al. 2009), (Bankole A. Johnson et
al. 2011), (Kenna, Zywiak, Swift, McGeary, Clifford, Shoaff, Vuittonet,
et al. 2014)) and/or had a positive drug screening test ((Sellers et al.
1994), (Bankole A. Johnson, Ait‐Daoud, and Prihoda 2000), (Bankole A.
Johnson, Roache, et al. 2000), (Ait-Daoud et al. 2001), (Myrick et al.
2008), (Corrêa Filho and Baltieri 2013)) were often excluded, except for
cannabis use in a few trials ((Dawes, Johnson, Ait-Daoud, et al. 2005),
(Myrick et al. 2008)). Receipt of alcohol use disorder treatment prior
to enrollment was an exclusion criteria; one study ((Sellers et al.
1994)) considered treatment over the previous 12 months, and 3 ((Bankole
A. Johnson, Cloninger, et al. 2000), (Ait-Daoud et al. 2001), (Dawes,
Johnson, Ait-Daoud, et al. 2005)) over the previous 30 days.

Some studies excluded participants who had been treated with stimulants,
sedatives, hypnotics or with treatment that could have an effect on
alcohol consumption or mood ((Corrêa Filho and Baltieri 2013), (Bankole
A. Johnson, Roache, et al. 2000), (Bankole A. Johnson, Ait‐Daoud, and
Prihoda 2000), (Ait-Daoud et al. 2001), (Dawes, Johnson, Ait-Daoud, et
al. 2005)). In one study ((Brown et al. 2021)), having been treated with
naltrexone, acamprosate, disulfiram or topiramate 2 weeks prior
inclusion, or current treatment with phenytoin, carbamazepine,
rifampicine, apomorphine or tramodol (due to potential interactions with
ondansetron) were exclusion criteria.

Psychiatric disorders were exclusion criteria in most studies ((Bankole
A. Johnson, Roache, et al. 2000), (Bankole A. Johnson, Ait‐Daoud, and
Prihoda 2000), (Ait-Daoud et al. 2001) (Kranzler et al. 2003), (Kenna et
al. 2009), (Bankole A. Johnson et al. 2011), (Kenna, Zywiak, Swift,
McGeary, Clifford, Shoaff, Fricchione, et al. 2014)). Two studies
considered only major diagnosis ((Myrick et al. 2008)) or clinically
significant disorders ((Corrêa Filho and Baltieri 2013) (Dawes, Johnson,
Ait-Daoud, et al. 2005)) as exclusion criteria. In the study of Sellers
\emph{et al.} ((Sellers et al. 1994)), a Montgomery/Asberg Depression
scale score below 15 and a Spielberg State-Trait anxiety inventory score
below 55 were required. In contrast, one study ((Brown et al. 2021))
enrolled only people with a concurrent psychiatric diagnosis.

Many studies also required good health at enrollment and notably
excluded frequent AUD comorbidities such as elevated bilirubin ((Bankole
A. Johnson, Roache, et al. 2000), (Kenna et al. 2009)), liver enzymes
((Dawes, Johnson, Ait-Daoud, et al. 2005), (Kenna et al. 2009)), (Brown
et al. 2021)), liver cirrhosis ((Corrêa Filho and Baltieri 2013), (Brown
et al. 2021)) or severe alcohol withdrawal ((Bankole A. Johnson,
Ait‐Daoud, and Prihoda 2000), (Bankole A. Johnson, Roache, et al. 2000),
(Dawes, Johnson, Ait-Daoud, et al. 2005)) ((Brown et al. 2021)).

Considering participants subtypes, 9 studies ((Bankole A. Johnson,
Roache, et al. 2000), (Bankole A. Johnson, Ait‐Daoud, and Prihoda 2000),
(Ait-Daoud et al. 2001), (Ait‐Daoud et al. 2001), (Bankole A. Johnson et
al. 2002), (Bankole A. Johnson et al. 2003), (Kranzler et al. 2003),
(Roache et al. 2008)), (Brown et al. 2021)) focused on clinical
characteristics based on age of onset, before or after the age of 25
(EOA and LOA, while 6 studies examined patient subtypes according to
their genotypes ((Bankole A. Johnson et al. 2011), (Seneviratne and
Johnson 2012), (Bankole A. Johnson et al. 2013), (Kenna, Zywiak, Swift,
McGeary, Clifford, Shoaff, Fricchione, et al. 2014), (Kenna, Zywiak,
Swift, McGeary, Clifford, Shoaff, Vuittonet, et al. 2014), (Hou et al.
2015)). The most frequent investigated gene was SLC6A4, coding for the
serotonin transporter that contains a polymorphism in the promoter
region, the 5-HTT-linked polymorphic region, with a "short" (S) and
"long" (L). Individuals possessing two long alleles (L/L) have been
found to respond differently to serotoninergic treatments than those
having either one (L/S) or two (S/S) short alleles. Another allele,
rs1042173-TT, also on the serotonine transporter gene predicted a better
response to ondansetron on alcohol use. Other genes of interest were
HTR3A and HTR3B, which regulate the 5HT3 receptor had polymorphisms
which influenced response to ondansetron (rs1150226-AG and rs1176713-GG
in HTR3A and rs17614942-AC in HTR3B).

Ondansetron dosage ranged from 1 µg/kg bid (twice a day) to 16 mg per
day with the most frequent dosage being 4 µg/kg bid. One study ((Brown
et al. 2021)) used a flexible dosage according to treatment response
that could range from 0.5 to 4 mg bid (with a mean dose of 3.24 ± 2.64
mg/day).

Four studies ((Bankole A. Johnson, Ait‐Daoud, and Prihoda 2000),
(Ait-Daoud et al. 2001), (Ait‐Daoud et al. 2001), (Myrick et al. 2008)),
involving 127 patients, used ondansetron in combination with naltrexone
(50 mg/d). Three studies (n=97 patients) compared ondansetron (0.5 mg/d)
to sertraline (200 mg/d) ((Kenna et al. 2009), (Kenna, Zywiak, Swift,
McGeary, Clifford, Shoaff, Fricchione, et al. 2014), (Kenna, Zywiak,
Swift, McGeary, Clifford, Shoaff, Vuittonet, et al. 2014)). Treatment
duration ranged from 8 days to 11 weeks. Pill count or riboflavin dosage
were used for assessing treatment compliance.

\subsubsection{3.3.2 Treatment outcomes}\label{treatment-outcomes}

Efficacy was most often assessed by evaluating the number of standard
drinks and derived variables such as defined by the Alcohol Timeline
Followback (TLFB) method (Sobell and Sobell 1992). Drinking outcomes
were drinks per day (DD), drinks per drinking day (DDD), percentage of
day abstinent (PDA), heavy drinking days (days with more than 5 drinks
per day), percentage of heavy drinking day (PHDD). Standard drink
definition varied across the studies, either 12 g ((Bankole A. Johnson,
Roache, et al. 2000)Bankole A. Johnson et al. (2011)), 13 g ((Sellers et
al. 1994) or 14 g (Corrêa Filho and Baltieri 2013)) of pure ethanol.
Furthermore, objective measures of alcohol use were used in 8 studies
with either carbohydrate deficient transferrin (CDT) ((Bankole A.
Johnson, Roache, et al. 2000), (Ait‐Daoud et al. 2001), (Kranzler et al.
2003),Brown et al. (2021)), or the volume of alcohol consumed during
self-administration ((Kenna et al. 2009), (Kenna, Zywiak, Swift,
McGeary, Clifford, Shoaff, Fricchione, et al. 2014), (Kenna, Zywiak,
Swift, McGeary, Clifford, Shoaff, Vuittonet, et al. 2014)), and
\(\gamma\)-glutamyltransferase in one study ((Brown et al. 2021)). X
studies examined the effects of ondansetron on alcohol craving. Some
studies ((Ait-Daoud et al. 2001), (Bankole A. Johnson et al. 2002)Myrick
et al. (2008)) evaluated alcohol craving, either with a visual
analogical scale or with the obsessive compulsive drinking scale (OCDS
(Anton, Moak, and Latham 1996)) and the Penn Alcohol Craving Scale
(PACS). ((Brown et al. 2021)) One study ((Dawes, Johnson, Ma, et al.
2005)) used the Adolescent Obsessive--Compulsive Drinking Scale
(A-OCDS). One study ((Myrick et al. 2008)) used functional magnetic
resonance imaging to determine ventral striatum activation during
cue-exposure, in addition to cue-induced craving assessment. Finally,
two studies examined the effcts of ondasetron on mood. One study
((Bankole A. Johnson et al. 2003)) used the Profile of Mood States
(Mcnair, Lorr, and Droppleman 1989) to evaluate attenuation of mood
disturbances, and another study ((Brown et al. 2021)) the Hamilton
Rating Scale for Depression (HRSD), the Young Mania Rating Scale (YMRS),
and the Inventory of Depressive Symptomatology--Self-report (IDS-SR).

\subsubsection{3.3.3 Effects of ondansetron on alcohol use
reduction}\label{effects-of-ondansetron-on-alcohol-use-reduction}

Eleven studies, summarized in \textbf{Table 2}, evaluated the impact of
ondansetron, alone or in combination with naltrexone, on alcohol use.
The main outcomes were mostly self-reported changes in alcohol
consumption according to the TLFB method, plasma CDT ((Bankole A.
Johnson, Roache, et al. 2000), (Ait‐Daoud et al. 2001), (Kranzler et al.
2003), (Brown et al. 2021)), GGT ((Brown et al. 2021)) or the volume of
alcohol consumed during self-administration ((Kenna et al. 2009),
(Kenna, Zywiak, Swift, McGeary, Clifford, Shoaff, Fricchione, et al.
2014), (Kenna, Zywiak, Swift, McGeary, Clifford, Shoaff, Vuittonet, et
al. 2014)).

In a sample of 71 males suffering from alcohol dependence (DSM-III-TR),
Sellers \emph{et al.} (Sellers et al. 1994) showed a significant impact
of ondansetron on alcohol reduction, when patients drinking more than 10
drinks per day at baseline were excluded from the analysis, with the
lower ondansetron dose (0,25 mg/d) producing the greatest reduction from
baseline.

One randomized control trial (Bankole A. Johnson, Ait‐Daoud, and Prihoda
2000) (n = 20) examined the impact of a combination of ondansetron and
naltrexone only among EOA participants. Results found significant effect
on reduction of drinks per day, drinks per drinking day, as well as a
trend in reducing the percentage of abstinent days, compared to placebo.

A subsequent analysis of the sample by Ait-Daoud \emph{et al.} showed
that the combination of ondansetron and naltrexone was associated with
significantly lower CDT levels (Ait‐Daoud et al. 2001).

In the only trial taking place outside of North America, Corrêa Filho
\emph{et al.} (Corrêa Filho and Baltieri 2013) showed a significant
reduction of heavy drinking days (7,8 \% vs 11,7\%, p=0.02) but not of
other alcohol-related clinical outcomes.

\subsubsection{3.3.4. Effects of ondansetron on alcohol use reduction
according patient
subtypes}\label{effects-of-ondansetron-on-alcohol-use-reduction-according-patient-subtypes}

\textbf{EOA vs LOA}

In a later trial (Johnson \emph{et al.} 2000b (Bankole A. Johnson,
Roache, et al. 2000)), Johnson et al.~examined ondansetron efficacy
according to patient subtypes (EOA vs LOA) among a sub-sample of 271
patients. Ondansetron was found to be significantly more effective than
placebo in reducing alcohol consumption among EOAs but not LOAs.
Ondansetron at 4 µg/kg \emph{b.i.d.} was more effective than placebo on
drinks per day (1.56 vs 3.30, p = 0.01), drinks per drinking day (4.28
vs 6.90, p = 0.004), percentage of day abstinent (70.10 vs 50.20, p =
0.02) and mean log CDT ratio (-0.19 vs 0.12, p = 0.01). Among EOAs, all
other dosages were superior to placebo on the two first criteria. These
results were subsequently replicated by Kranzler \emph{et al.} in 2003
(Kranzler et al. 2003), who showed a significant reduction (compared to
baseline) in most alcohol-related measures (drinks per day, drinks per
drinking days, DrinC total score) among EOAs and LOAs who received
ondansetron (4 µg/kg \emph{b.i.d.}). A significant difference was also
found according to patient subtypes, with significant greater decrease
of drinks per day, drinks per drinking day and DrinC total score among
EOAs. Changes in the level of carbohydrate-deficient transferrin were
consistent with changes in self-reported drinking behavior.

\textbf{Patient genotypes}

In a randomized controlled trial, Johnson \emph{et al.} (Bankole A.
Johnson et al. 2011) randomized 283 patients by genotype in the
5\textquotesingle-regulatory region of the 5-HTT gene (LL/LS/SS).
Individuals with the L/L genotype receiving ondansetron significantly
reduced their alcohol consumption, measured by drinks per drinking day
and percentage of days abstinent as compared to placebo (respectively
-1.62, p = 0.007 and 11.27\%, p = 0.023).

Two other studies examined treatment response according to the L/L
genotype using a self-administration experiment. The study of Kenna
\emph{et al.} (Kenna et al. 2009), among 15 non-treatment seeking
individuals, found that participants with the L/L genotype who were
administered ondansetron (4 µg/kg \emph{b.i.d.}) during 3 weeks drank
significantly less alcohol during an alcohol self-administration
procedure, than their counterparts being treated with sertaline (200 mg
per day). In another clinical trial involving 77 participants, the same
authors provided limited support that ondansetron may reduce drinking in
non treatment-seeking individuals with the LL genotype ((Kenna, Zywiak,
Swift, McGeary, Clifford, Shoaff, Vuittonet, et al. 2014)). Further
analysis ((Kenna, Zywiak, Swift, McGeary, Clifford, Shoaff, Fricchione,
et al. 2014)) pinpointed gender differences as L/L women treated with
ondansetron and S/L or S/S women treated with sertraline had
significantly less drinks per drinking days and drank during alcohol
self-administration experiment.

\textbf{Alcohol and bipolar disorder participants}

Recently, Sherwood \emph{et al.} (Brown et al. 2021) evaluated the
efficacy of ondansetron in 70 patients suffering from both alcohol use
disorder and bipolar disorder. Results showed a trend in greater
reduction of drinking as measured by the TLFB method that failed to
reach significance.

\subsubsection{3.3.5. Effects on craving and cue-induced
craving}\label{effects-on-craving-and-cue-induced-craving}

Four studies (\textbf{Table 3}) examined the impact of ondansetron on
craving.

An analysis of the study population of Johnson \emph{et al.} 2000a
((Bankole A. Johnson, Ait‐Daoud, and Prihoda 2000)) found that the
combination of ondansetron and naltrexone was significantly better than
placebo in reducing craving among EOAs (Ait-Daoud et al. 2001). Johnson
\emph{et al.} 2002 (Bankole A. Johnson et al. 2002) showed that
ondansetron at 4 µg/kg \emph{b.i.d.} was associated with a significant
reduction in craving (measured by visual analog scale) compared to the
placebo group, but only in EOAs. In contrast, craving was significantly
increased by ondansetron at 1 µg/kg \emph{b.i.d.} among LOAs.

In a BOLD-MRI laboratory study, Myrick \emph{et al.} (Myrick et al.
2008) examined ventral striatum activation of AUD patients during
cue-exposure, according to treatment group (naltrexone, ondansetron, a
combination of both or placebo), and compared to social-drinkers.
Ventral striatum activation was significantly reduced in the naltrexone,
combination and social-drinker groups, and this reduction was correlated
with reduction in craving scores. Ondansetron alone was not
significantly effective in the reduction of ventral striatum activation
nor craving. There was however a trend in the reduction of both.

The study of Sherwood \emph{et al.} (Brown et al. 2021) found no
differences between ondansetron and placebo groups, among patients with
both bipolar disorder and AUD.

\subsubsection{3.3.6. Effects on mood
disturbances}\label{effects-on-mood-disturbances}

Two studies investigated the impact of ondansetron medication on mood
(\textbf{Table 4}). One study ((Bankole A. Johnson et al. 2003)) showed
that ondansetron significantly reduced mood disturbances as measured by
the POMS scale among EAOs only. Sherwood \emph{et al.} (Brown et al.
2021) showed that ondansetron was significantly more efficacious than
placebo in the reduction of HRSD scores, but not of YMRS or IDS-SR
scores, among patients suffering from both bipolar disorder and AUD.

\subsubsection{3.3.7. Predictors of treatment
outcomes}\label{predictors-of-treatment-outcomes}

Finally, six studies, presented in \textbf{Table 5,} provided data on
predictors of treatment outcomes.

As ondansetron had previously been found useful in early onset
alcoholism, Dawes \emph{et al.} (Dawes, Johnson, Ait-Daoud, et al. 2005)
evaluated ondansetron among adolescents in a prospective, open-label
trial, which showed that ondansetron was safe and well tolerated in
adolescents with alcohol dependence. In a subsequent study (Dawes,
Johnson, Ma, et al. 2005), they found that reduction of drinking
correlated with reduction in craving. Roache \emph{et al.} 2008 (Roache
et al. 2008) compared the prediction capabilities of the EOA/LOA
typology to the type A/type B typology described by Babor \emph{et al.}
1992 (Babor et al. 1992) (derived from Type I/II description by
Cloninger \emph{et al.} (Cloninger 1987)). The A/B typology better
described baseline severity of alcohol dependence but treatment response
to ondansetron was significantly better predicted by the EOA/LOA
typology.

Seneviratne \emph{et al.} (Seneviratne and Johnson 2012) produced some
evidence that 5'HTTPLR mRNA levels could be used as biomarker to
evaluate treatment effectiveness in L/L-subjects treated with
ondansetron.

Two studies analysed the population of Johnson \emph{et al.} 2011
(Bankole A. Johnson et al. 2011) and identified genotypes predicting
treatment success. Johnson \emph{et al.} 2013 (Bankole A. Johnson et al.
2013) found 5 genotypes that are highly prevalent in the general
population and that predicted efficacy of ondansetron. Finally, Hou
\emph{et al.} (Hou et al. 2015) further worked on identifying ways of
predicting effectiviness of ondansetron and found that data mining
approaches, such as interaction trees and virtual twins could simplify
subgroup identification while limiting statistical errors.

\section{Discussion}\label{discussion}

To our knowledge, the present review is the only systematic review
assessing efficacy of including ondansetron for the treatment of alcohol
use disorder. A subsequent research identified 5 systematic reviews in
the last 10 years that included the keywords `ondansetron' and `alcohol
use disorder' or `alcoholism'. Bauer \emph{et al.} 2015 (Bauer et al.
2015) focused on the influence of serotonergic gene variation in
substance use pharmacotherapy and included four out of the 21 studies
presented here. Naglich \emph{et al.} 2018 (Naglich et al. 2017) focused
on combined pharmacotherapy for the treament of alcohol use and thus
included 2 studies involving ondansetron and naltrexone. Cservenka
\emph{et al.} 2017 (Cservenka, Yardley, and Ray 2016) focused on
pharmacogenetics and the implication of ethnic diversity in the
treatment of AUD and included 2 papers. Castrén \emph{et al.} 2019
(Castrén, Mäkelä, and Alho 2019) focused on the recent findings in AUD
pharmacotherapy and mentioned ondansetron without including any of the
clinical trials. Finally, Bharadwaj \emph{et al.} 2018 (Bharadwaj,
Selvakumar, and Kuppili 2018) focused on the pharmacotherapy for relapse
prevention in AUD in the Indian setting and also mentioned ondansetron
but didn't include any trial on this topic. One review (Thompson and
Kenna 2015) focuses on the role of the serotonin transporter gene in AUD
and thus cites 6 of the latest pharmacogenetics studies.

Most of the clinical trials described in this systematic review have
stringent inclusion criteria which greatly limits their external
validity. Particularly, patients suffering from dual diagnosis or
addicted to several substance (with the exception of nicotine) were
often excluded. The high dropout rate (mean dropout rate 35.4\%) could
impact the validity of the findings, but evidence to identify whether or
not dropout rate favors medication is lacking.

Finally, out of the seven registered trials that have no published
papers yet, three haven't had any updates for more than ten years
whereas one reported non significant results. This may pose a
publication bias that is to be taken in consideration.

\section{Conclusion}\label{conclusion}

Whereas growing evidence tends to suggest efficacy of ondansetron as a
treatment of alcohol use disorder in particular genetic subgroups,
further studies are needed to fully conclude. In particular, there is a
need for bigger studies evaluating long term changes in alcohol
consumption. These studies should also have less exclusion criteria to
maximize their external validity.

\section{References}\label{references}

\phantomsection\label{refs}
\begin{CSLReferences}{1}{0}
\bibitem[\citeproctext]{ref-ait-daoud2001CombiningOndansetronNaltrexone}
Ait‐Daoud, Nassima, Bankole A. Johnson, Martin Javors, John D. Roache,
and Nursen A. Zanca. 2001. {``Combining {Ondansetron} and {Naltrexone
Treats Biological Alcoholics}: {Corroboration} of {Self}‐{Reported
Drinking} by {Serum Carbohydrate Deficient Transferrin}, {A
Biomarker}.''} \emph{Alcoholism: Clinical and Experimental Research} 25
(6): 847--49. \url{https://doi.org/10.1111/j.1530-0277.2001.tb02289.x}.

\bibitem[\citeproctext]{ref-ait-daoud2001CombiningOndansetronNaltrexonea}
Ait-Daoud, Nassima, Bankole A. Johnson, Thomas J. Prihoda, and Irene D.
Hargita. 2001. {``Combining Ondansetron and Naltrexone Reduces Craving
Among Biologically Predisposed Alcoholics: Preliminary Clinical
Evidence.''} \emph{Psychopharmacology} 154 (1): 23--27.
\url{https://doi.org/10.1007/s002130000607}.

\bibitem[\citeproctext]{ref-ait-daoud2009CanSerotoninTransporter}
Ait-Daoud, Nassima, John D. Roache, Michael A. Dawes, Lei Liu, Xin-Qun
Wang, Martin A. Javors, Chamindi Seneviratne, and Bankole A. Johnson.
2009. {``Can Serotonin Transporter Genotype Predict Craving in
Alcoholism?''} \emph{Alcoholism, Clinical and Experimental Research} 33
(8): 1329--35. \url{https://doi.org/10.1111/j.1530-0277.2009.00962.x}.

\bibitem[\citeproctext]{ref-anton1996ObsessiveCompulsiveDrinking}
Anton, R. F., D. H. Moak, and P. K. Latham. 1996. {``The Obsessive
Compulsive Drinking Scale: {A} New Method of Assessing Outcome in
Alcoholism Treatment Studies.''} \emph{Archives of General Psychiatry}
53 (3): 225--31.
\url{https://doi.org/10.1001/archpsyc.1996.01830030047008}.

\bibitem[\citeproctext]{ref-babor1992TypesAlcoholicsEvidence}
Babor, Thomas F., Michael Hofmann, Frances K. DelBoca, Victor
Hesselbrock, Roger E. Meyer, Zelig S. Dolinsky, and Bruce Rounsaville.
1992. {``Types of {Alcoholics}, {I}: {Evidence} for an {Empirically
Derived Typology Based} on {Indicators} of {Vulnerability} and
{Severity}.''} \emph{Archives of General Psychiatry} 49 (8): 599--608.
\url{https://doi.org/10.1001/archpsyc.1992.01820080007002}.

\bibitem[\citeproctext]{ref-bauer2015SerotonergicGeneVariation}
Bauer, Isabelle E, David P Graham, Jair C Soares, and David A Nielsen.
2015. {``Serotonergic {Gene Variation} in {Substance Use
Pharmacotherapy}: {A Systematic Review}.''} \emph{Pharmacogenomics} 16
(11): 1305--12. \url{https://doi.org/10.2217/pgs.15.72}.

\bibitem[\citeproctext]{ref-bharadwaj2018PharmacotherapyRelapsePrevention}
Bharadwaj, Balaji, Nivedhitha Selvakumar, and PoojaPatnaik Kuppili.
2018. {``Pharmacotherapy for Relapse Prevention of Alcohol Use Disorder
in the {Indian} Setting: {A} Systematic Review.''} \emph{Industrial
Psychiatry Journal} 27 (2): 163.
\url{https://doi.org/10.4103/ipj.ipj_79_17}.

\bibitem[\citeproctext]{ref-brown2021RandomizedDoubleblindPlacebocontrolled}
Brown, E. Sherwood, Meagan McArdle, Jayme Palka, Collette Bice, Elena
Ivleva, Alyson Nakamura, Markey McNutt, Zena Patel, Traci Holmes, and
Shane Tipton. 2021. {``A Randomized, Double-Blind, Placebo-Controlled
Proof-of-Concept Study of Ondansetron for Bipolar and Related Disorders
and Alcohol Use Disorder.''} \emph{European Neuropsychopharmacology} 43
(February): 92--101.
\url{https://doi.org/10.1016/j.euroneuro.2020.12.006}.

\bibitem[\citeproctext]{ref-castren2019SelectingAppropriateAlcohol}
Castrén, Sari, Niklas Mäkelä, and Hannu Alho. 2019. {``Selecting an
Appropriate Alcohol Pharmacotherapy.''} \emph{Current Opinion in
Psychiatry} 32 (4): 266--74.
\url{https://doi.org/10.1097/YCO.0000000000000512}.

\bibitem[\citeproctext]{ref-cloninger1987SystematicMethodClinical}
Cloninger, C. R. 1987. {``A Systematic Method for Clinical Description
and Classification of Personality Variants. {A} Proposal.''}
\emph{Archives of General Psychiatry} 44 (6): 573--88.
\url{https://doi.org/10.1001/archpsyc.1987.01800180093014}.

\bibitem[\citeproctext]{ref-correafilho2013PilotStudyFulldose}
Corrêa Filho, João Maria, and Danilo Antonio Baltieri. 2013. {``A Pilot
Study of Full-Dose Ondansetron to Treat Heavy-Drinking Men Withdrawing
from Alcohol in {Brazil}.''} \emph{Addictive Behaviors} 38 (4):
2044--51. \url{https://doi.org/10.1016/j.addbeh.2012.12.018}.

\bibitem[\citeproctext]{ref-cservenka2016ReviewPharmacogeneticsAlcoholism}
Cservenka, Anita, Megan M. Yardley, and Lara A. Ray. 2016. {``Review:
{Pharmacogenetics} of Alcoholism Treatment: {Implications} of Ethnic
Diversity.''} \emph{The American Journal on Addictions} 26 (5): 516--25.
\url{https://doi.org/10.1111/ajad.12463}.

\bibitem[\citeproctext]{ref-dawes2005ProspectiveOpenlabelTrial}
Dawes, Michael A., Bankole A. Johnson, Nassima Ait-Daoud, Jennie Z. Ma,
and Jack R. Cornelius. 2005. {``A Prospective, Open-Label Trial of
Ondansetron in Adolescents with Alcohol Dependence.''} \emph{Addictive
Behaviors} 30 (6): 1077--85.
\url{https://doi.org/10.1016/j.addbeh.2004.10.011}.

\bibitem[\citeproctext]{ref-dawes2005ReductionsRelationsCraving}
Dawes, Michael A., Bankole A. Johnson, Jennie Z. Ma, Nassima Ait-Daoud,
Suzanne E. Thomas, and Jack R. Cornelius. 2005. {``Reductions in and
Relations Between {`Craving'} and Drinking in a Prospective, Open-Label
Trial of Ondansetron in Adolescents with Alcohol Dependence.''}
\emph{Addictive Behaviors} 30 (9): 1630--37.
\url{https://doi.org/10.1016/j.addbeh.2005.07.004}.

\bibitem[\citeproctext]{ref-grant1991BlockadeDiscriminativeStimulus}
Grant, Kathleen A., and James E. Barrett. 1991. {``Blockade of the
Discriminative Stimulus Effects of Ethanol with 5-{HT3} Receptor
Antagonists.''} \emph{Psychopharmacology} 104 (4): 451--56.
\url{https://doi.org/10.1007/BF02245648}.

\bibitem[\citeproctext]{ref-hagan1987Effect5HT3Receptor}
Hagan, Russell M., Arthur Butler, Julia M. Hill, Christopher C. Jordan,
Simon J. Ireland, and Michael B. Tyers. 1987. {``Effect of the 5-{HT3}
Receptor Antagonist, {GR38032F}, on Responses to Injection of a
Neurokinin Agonist into the Ventral Tegmental Area of the Rat Brain.''}
\emph{European Journal of Pharmacology} 138 (2): 303--5.
\url{https://doi.org/10.1016/0014-2999(87)90450-X}.

\bibitem[\citeproctext]{ref-hou2015SubgroupIdentificationPersonalized}
Hou, Jue, Chamindi Seneviratne, Xiaogang Su, Jeremy Taylor, Bankole
Johnson, Xin‐Qun Wang, Heping Zhang, Henry R. Kranzler, Joseph Kang, and
Lei Liu. 2015. {``Subgroup {Identification} in {Personalized Treatment}
of {Alcohol Dependence}.''} \emph{Alcoholism: Clinical and Experimental
Research} 39 (7): 1253--59. \url{https://doi.org/10.1111/acer.12759}.

\bibitem[\citeproctext]{ref-johnson1993AttenuationAlcoholinducedMood}
Johnson, B. A., G. M. Campling, P. Griffiths, and P. J. Cowen. 1993.
{``Attenuation of Some Alcohol-Induced Mood Changes and the Desire to
Drink by 5-{HT3} Receptor Blockade: A Preliminary Study in Healthy Male
Volunteers.''} \emph{Psychopharmacology} 112 (1): 142--44.
\url{https://doi.org/10.1007/bf02247375}.

\bibitem[\citeproctext]{ref-johnson2003OndansetronReducesMood}
Johnson, Bankole A., Nassima Ait‐Daoud, Jennie Z. Ma, and Yanmei Wang.
2003. {``Ondansetron {Reduces Mood Disturbance Among Biologically
Predisposed}, {Alcohol}‐{Dependent Individuals}.''} \emph{Alcoholism:
Clinical and Experimental Research} 27 (11): 1773--79.
\url{https://doi.org/10.1097/01.ALC.0000095635.46911.5D}.

\bibitem[\citeproctext]{ref-johnson2000CombiningOndansetronNaltrexone}
Johnson, Bankole A., Nassima Ait‐Daoud, and Thomas J. Prihoda. 2000.
{``Combining {Ondansetron} and {Naltrexone Effectively Treats
Biologically Predisposed Alcoholics}: {From Hypotheses} to {Preliminary
Clinical Evidence}.''} \emph{Alcoholism: Clinical and Experimental
Research} 24 (5): 737--42.
\url{https://doi.org/10.1111/j.1530-0277.2000.tb02048.x}.

\bibitem[\citeproctext]{ref-johnson2011PharmacogeneticApproachSerotonin}
Johnson, Bankole A., Nassima Ait-Daoud, Chamindi Seneviratne, John D.
Roache, Martin A. Javors, Xin-Qun Wang, Lei Liu, J. Kim Penberthy, Carlo
C. DiClemente, and Ming D. Li. 2011. {``Pharmacogenetic {Approach} at
the {Serotonin Transporter Gene} as a {Method} of {Reducing} the
{Severity} of {Alcohol Drinking}.''} \emph{American Journal of
Psychiatry} 168 (3): 265--75.
\url{https://doi.org/10.1176/appi.ajp.2010.10050755}.

\bibitem[\citeproctext]{ref-johnson2000AgeOnsetDiscriminator}
Johnson, Bankole A., C. Robert Cloninger, John D. Roache, Patrick S.
Bordnick, and Pedro Ruiz. 2000. {``Age of {Onset} as a {Discriminator
Between Alcoholic Subtypes} in a {Treatment}‐{Seeking Outpatient
Population}.''} \emph{The American Journal on Addictions} 9 (1): 17--27.
\url{https://doi.org/10.1080/10550490050172191}.

\bibitem[\citeproctext]{ref-johnson2002OndansetronReducesCraving}
Johnson, Bankole A., John D. Roache, Nassima Ait-Daoud, Nursen A. Zanca,
and Madeline Velazquez. 2002. {``Ondansetron Reduces the Craving of
Biologically Predisposed Alcoholics.''} \emph{Psychopharmacology} 160
(4): 408--13. \url{https://doi.org/10.1007/s00213-002-1002-9}.

\bibitem[\citeproctext]{ref-johnson2000OndansetronReductionDrinking}
Johnson, Bankole A., John D. Roache, Martin A. Javors, Carlo C.
DiClemente, Claude Robert Cloninger, Thomas J. Prihoda, Patrick S.
Bordnick, Nassima Ait-Daoud, and Julie Hensler. 2000. {``Ondansetron for
{Reduction} of {Drinking Among Biologically Predisposed Alcoholic
Patients}.''} \emph{JAMA} 284 (8): 963.
\url{https://doi.org/10.1001/jama.284.8.963}.

\bibitem[\citeproctext]{ref-johnson2013DeterminationGenotypeCombinations}
Johnson, Bankole A., Chamindi Seneviratne, Xin-Qun Wang, Nassima
Ait-Daoud, and Ming D. Li. 2013. {``Determination of {Genotype
Combinations That Can Predict} the {Outcome} of the {Treatment} of
{Alcohol Dependence Using} the 5-{HT}{\textsubscript{3}} {Antagonist
Ondansetron}.''} \emph{American Journal of Psychiatry} 170 (9):
1020--31. \url{https://doi.org/10.1176/appi.ajp.2013.12091163}.

\bibitem[\citeproctext]{ref-kenna2009WithinGroupDesignNontreatment}
Kenna, George A., William H. Zywiak, John E. McGeary, Lorenzo Leggio,
Chinatsu McGeary, Shirley Wang, Andrea Grenga, and Robert M. Swift.
2009. {``A {Within}‐{Group Design} of {Nontreatment Seeking} 5‐{HTTLPR
Genotyped Alcohol}‐{Dependent Subjects Receiving Ondansetron} and
{Sertraline}.''} \emph{Alcoholism: Clinical and Experimental Research}
33 (2): 315--23. \url{https://doi.org/10.1111/j.1530-0277.2008.00835.x}.

\bibitem[\citeproctext]{ref-kenna2014OndansetronSertralineMay}
Kenna, George A., William H. Zywiak, Robert M. Swift, John E. McGeary,
James S. Clifford, Jessica R. Shoaff, Samuel Fricchione, et al. 2014.
{``Ondansetron and Sertraline May Interact with 5-{HTTLPR} and {DRD4}
Polymorphisms to Reduce Drinking in Non-Treatment Seeking
Alcohol-Dependent Women: {Exploratory} Findings.''} \emph{Alcohol} 48
(6): 515--22. \url{https://doi.org/10.1016/j.alcohol.2014.04.005}.

\bibitem[\citeproctext]{ref-kenna2014OndansetronReducesNaturalistic}
Kenna, George A., William H. Zywiak, Robert M. Swift, John E. McGeary,
James S. Clifford, Jessica R. Shoaff, Cynthia Vuittonet, et al. 2014.
{``Ondansetron {Reduces Naturalistic Drinking} in {Nontreatment-Seeking
Alcohol-Dependent Individuals} with the {LL} 5'-{HTTLPR Genotype}: {A
Laboratory Study}.''} \emph{Alcoholism: Clinical and Experimental
Research} 38 (6): 1567--74. \url{https://doi.org/10.1111/acer.12410}.

\bibitem[\citeproctext]{ref-kranzler2003EffectsOndansetronEarly}
Kranzler, Henry R., Amira Pierucci‐Lagha, Richard Feinn, and Carlos
Hernandez‐Avila. 2003. {``Effects of {Ondansetron} in {Early}‐ {Versus
Late}‐{Onset Alcoholics}: {A Prospective}, {Open}‐{Label Study}.''}
\emph{Alcoholism: Clinical and Experimental Research} 27 (7): 1150--55.
\url{https://doi.org/10.1097/01.ALC.0000075547.77464.76}.

\bibitem[\citeproctext]{ref-mcguinness2020RiskofbiasVISualizationRobvis}
McGuinness, Luke A., and Julian P. T. Higgins. 2020. {``Risk‐of‐bias
{VISualization} (Robvis): {An R} Package and {Shiny} Web App for
Visualizing Risk‐of‐bias Assessments.''} \emph{Research Synthesis
Methods} 12 (1): 55--61. \url{https://doi.org/10.1002/jrsm.1411}.

\bibitem[\citeproctext]{ref-Mcnair1989ProfileOM}
Mcnair, Douglas M., Maurice Lorr, and Leo F. Droppleman. 1989.
{``Profile of Mood States ({POMS}).''} In.
\url{https://api.semanticscholar.org/CorpusID:142896988}.

\bibitem[\citeproctext]{ref-myrick2008EffectNaltrexoneOndansetron}
Myrick, Hugh, Raymond F Anton, Xingbao Li, Scott Henderson, Patrick K
Randall, and Konstantin Voronin. 2008. {``Effect of {Naltrexone} and
{Ondansetron} on {Alcohol Cue}--{Induced Activation} of the {Ventral
Striatum} in {Alcohol-Dependent People}.''} \emph{ARCH GEN PSYCHIATRY}
65 (4): 10.

\bibitem[\citeproctext]{ref-naglich2017SystematicReviewCombined}
Naglich, Andrew C., Austin Lin, Sidarth Wakhlu, and Bryon H. Adinoff.
2017. {``Systematic {Review} of {Combined Pharmacotherapy} for the
{Treatment} of {Alcohol Use Disorder} in {Patients Without Comorbid
Conditions}.''} \emph{CNS Drugs} 32 (1): 13--31.
\url{https://doi.org/10.1007/s40263-017-0484-2}.

\bibitem[\citeproctext]{ref-peterson2011NewcastleOttawaScaleNOS}
Peterson, J, V Welch, M Losos, and PJOOHRI Tugwell. 2011. {``The
{Newcastle-Ottawa} Scale ({NOS}) for Assessing the Quality of
Nonrandomised Studies in Meta-Analyses.''} \emph{Ottawa: Ottawa Hospital
Research Institute}, 1--12.

\bibitem[\citeproctext]{ref-roache2008PredictionSerotonergicTreatment}
Roache, John D., Yanmei Wang, Nassima Ait‐Daoud, and Bankole A. Johnson.
2008. {``Prediction of {Serotonergic Treatment Efficacy Using Age} of
{Onset} and {Type A}/{B Typologies} of {Alcoholism}.''}
\emph{Alcoholism: Clinical and Experimental Research} 32 (8): 1502--12.
\url{https://doi.org/10.1111/j.1530-0277.2008.00717.x}.

\bibitem[\citeproctext]{ref-sellers1994ClinicalEfficacy5HT3}
Sellers, Edward M., Tony Toneatto, Myroslava K. Romach, Gail R. Somer,
Linda C. Sobell, and Mark B. Sobell. 1994. {``Clinical {Efficacy} of the
5‐{HT}{\textsubscript{3}} {Antagonist Ondansetron} in {Alcohol Abuse}
and {Dependence}.''} \emph{Alcoholism: Clinical and Experimental
Research} 18 (4): 879--85.
\url{https://doi.org/10.1111/j.1530-0277.1994.tb00054.x}.

\bibitem[\citeproctext]{ref-seneviratne2012SerotoninTransporterGenomic}
Seneviratne, Chamindi, and Bankole A. Johnson. 2012. {``Serotonin
{Transporter Genomic Biomarker} for {Quantitative Assessment} of
{Ondansetron Treatment Response} in {Alcoholics}.''} \emph{Frontiers in
Psychiatry} 3: 23. \url{https://doi.org/10.3389/fpsyt.2012.00023}.

\bibitem[\citeproctext]{ref-sobell1992TimelineFollowBack}
Sobell, Linda C., and Mark B. Sobell. 1992. {``Timeline
{Follow-Back}.''} In \emph{Measuring {Alcohol Consumption}:
{Psychosocial} and {Biochemical Methods}}, edited by Raye Z. Litten and
John P. Allen, 41--72. Totowa, NJ: Humana Press.
\url{https://doi.org/10.1007/978-1-4612-0357-5_3}.

\bibitem[\citeproctext]{ref-sterne2019RoBRevisedTool}
Sterne, Jonathan A C, Jelena Savović, Matthew J Page, Roy G Elbers,
Natalie S Blencowe, Isabelle Boutron, Christopher J Cates, et al. 2019.
{``{RoB} 2: A Revised Tool for Assessing Risk of Bias in Randomised
Trials.''} \emph{BMJ} 366 (August): l4898.
\url{https://doi.org/10.1136/bmj.l4898}.

\bibitem[\citeproctext]{ref-thompson2015VariationSerotoninTransporter}
Thompson, Miles D., and George A. Kenna. 2015. {``Variation in the
{Serotonin Transporter Gene} and {Alcoholism}: {Risk} and {Response} to
{Pharmacotherapy}.''} \emph{Alcohol and Alcoholism} 51 (2): 164--71.
\url{https://doi.org/10.1093/alcalc/agv090}.

\bibitem[\citeproctext]{ref-varma1994CorrelatesEarlyLateonset}
Varma, Vijoy K., Debasish Basu, Anil Malhotra, Avneet Sharma, and
Surendra K. Mattoo. 1994. {``Correlates of Early- and Late-Onset Alcohol
Dependence.''} \emph{Addictive Behaviors} 19 (6): 609--19.
\url{https://doi.org/10.1016/0306-4603(94)90016-7}.

\end{CSLReferences}

\end{document}
